\documentclass{micro-econ-thesis}

% ----------------------------------------------------------------------------
% PACKAGES
% ----------------------------------------------------------------------------
\usepackage[utf8]{inputenc} 
\usepackage[T1]{fontenc}
\usepackage{lmodern}
\usepackage[english]{babel}
\usepackage{csquotes}
\usepackage{booktabs}
\usepackage{amsmath,amssymb}
\usepackage{enumitem}
\usepackage{appendix}
\usepackage{graphicx}
\usepackage{pdfpages}
\usepackage{setspace}
\usepackage{xcolor, colortbl} % colors and table coloring
\usepackage{hyperref}
\usepackage{fancyhdr}

% ----------------------------------------------------------------------------
% COLORS
% ----------------------------------------------------------------------------
\definecolor{tableheader}{HTML}{EED799} % light gray header background
\definecolor{rowcolor}{HTML}{F9F9F9}     % alternate row color

% ----------------------------------------------------------------------------
% BIBLIOGRAPHY
% ----------------------------------------------------------------------------
\usepackage[giveninits=true, uniquename=init, authordate]{biblatex-chicago}
\addbibresource{bibliography.bib}

% ----------------------------------------------------------------------------
% HEADER
% ----------------------------------------------------------------------------
\fancyhf{}
\fancyhead[R]{\small\thepage}
\fancyhead[L]{\small\nouppercase{\leftmark}}

% ----------------------------------------------------------------------------
% TEAM MEMBERS (custom macro for title page)
% ----------------------------------------------------------------------------
\thesisMembers{%
Amaan Parashar : 2023UCS0081\\
Nikhat Singla : 2023UCS0103\\
Patel Pritkumar Nareshbhai : 2023UCS0106\\
Ujjwal Vishwakarma : 2023UCS0116\\
Yashan Garg : 2023UCS0120
}

\begin{document}
% ----------------------------------------------------------------------------
% Details for the titlepage
% ----------------------------------------------------------------------------
\thesisTitle{Ticket Booking System for Delhi Metro Service}
\thesisType{End-Term Project}
\thesisAuthor{Group 4}
\thesisDate{\today}

% Print titlepage
\thesisMakeTitleEN

% ----------------------------------------------------------------------------
% Table of contents
% ----------------------------------------------------------------------------
\cleardoublepage

\tableofcontents

% ----------------------------------------------------------------------------
% List of figures/tables
% ----------------------------------------------------------------------------

\cleardoublepage

% --------------------------

% ----------------------------------------------------------------------------
% Contents
% ----------------------------------------------------------------------------
\cleardoublepage
\pagestyle{fancy}
\pagenumbering{arabic}
\setcounter{page}{1}
% Contents
\onehalfspacing % for linespacing 1.5, you can turn it off with \singlespacing, e.g. for quotes or tables with multiline cells


\section{Introduction}
\label{sec:first}

The goal is to design a comprehensive database schema for a modern metro ticket booking system. This system must manage the entire lifecycle of a passenger's interaction, from registering an account to completing a journey.

The core problem is to model and store data related to several key functions:

\textbf{1. User Management:} Securely handling passenger accounts, credentials, and personal details.

\textbf{2. Payment \& Transactions:} Reliably processing various financial activities, including ticket purchases and smart card recharges.

\textbf{3. Smart Card Operations:} Managing the balance, status, and recharge history of physical or digital metro cards.

\textbf{4. Ticketing:} Generating unique, valid tickets (e.g., QR codes) for travel based on a booking.

\textbf{5. Journey Validation:} Tracking the passenger's actual travel by recording entry (tap-in) and exit (tap-out) events at station gates.

\textbf{6. Network \& Fares:} Defining the physical metro network (stations, lines) and the travel fares between metro stations.

The database must ensure data integrity (e.g., a ticket cannot be used for two journeys), support specialization (e.g., a 'booking' is a specific type of 'transaction'), and model the real-world relationships between these concepts.

\newpage

\section{Ideation}
\label{sec:second}
To address the problem statement, the system was decomposed into the following fundamental entities, each representing a distinct real-world or logical concept:

\begin{itemize}
    \item \textbf{User:} The passenger or customer.
    \item \textbf{SmartCard:} The reusable rechargable card for metro travels.
    \item \textbf{Station, Line, StationLine:} The physical metro network.
    \item \textbf{Transaction:} The central record for any financial event (the "base" entity).
    \item \textbf{Booking \& Recharge:} Specialized types of transactions.
    \item \textbf{Payment:} The record of how a transaction was settled.
    \item \textbf{Ticket:} The single-use authorization for travel.
    \item \textbf{Journey:} The actual trip taken using a ticket.
    \item \textbf{Fare:} The store of fare price between any two metro stations.
\end{itemize}

\textbf{Key Design Justifications :-}

\vspace{5pt}                           
During the modeling process, several key decisions were made to ensure the system is flexible, scalable, and adheres to good database design principles.
\vspace{5pt} 

\textbf{1. Why have both SmartCard and Ticket?}\newline
This separation is crucial for flexibility and reflects two different real-world concepts:
\begin{enumerate}
    \renewcommand{\labelenumi}{\roman{enumi}.}
    \item SmartCard is a Payment Method (a Stored-Value Rechargable Wallet):
    \begin{itemize}
        \item Its primary role is to store monetary value.
        \item It has a Balance that users can Recharge.
        \item It is reusable and persistent. A user owns a card and tops it up over time.
        \item In this system, it acts as an internal wallet, which can be used to pay for a booking.
    \end{itemize}

    \item Ticket is a Travel Authorization (a Proof of Purchase):
    \begin{itemize}
        \item Its primary role is to grant access for a single, specific journey.
        \item It is generated as the result of a Booking.
        \item It is ephemeral and single-use. It contains the QRCode Data, is valid for a specific time, and is consumed by a Journey.
    \end{itemize}

    \item Why separate them?
    \begin{itemize}
        \item A Tourist/Infrequent User can pay for a Booking directly with a credit card. They never need a SmartCard.
        \item A regular commuter maintains a SmartCard with a minimum Rs. 50 Balance and use the app to make a Booking.
        \item It would be impossible to support users who just want to buy a single ticket without registering for a smart card.
    \end{itemize}
    
\end{enumerate}

\textbf{2. Why separate Transaction, Payment, Booking, and Recharge?}\newline
This is a classic example of specialization and separation of concerns, which makes the system much cleaner.

\begin{enumerate}
    \renewcommand{\labelenumi}{\roman{enumi}.}
    \item Transaction is the "Base" Entity:
    \begin{itemize}
        \item It acts as the central journal for all financial events.
        \item It stores the common data for any event: UserID, Amount, Status, and Timestamp.
        \item It is for internal use in the organization
    \end{itemize}


    \item Booking and Recharge are "Specialized" Entities:
    \begin{itemize}
        \item A Booking is a transaction with a purpose of traveling from a Source to a Destination.
        \item A Recharge is a transaction with a purpose of adding value to a CardID.
    \end{itemize}


    \item Payment is the "Settlement" Entity:
    \begin{itemize}
        \item A Transaction record is created (e.g., status "Pending") before the payment is even attempted.
        \item The Payment entity stores the operational details and the payment's specific status.
        \item Payment need not be initiated for every transaction (e.g. Rs. 0 transactions)
        \itme It is external to the organization(e.g. UPI gateway)
    \end{itemize}

    \item Why separate Transaction and Payment?
    \begin{itemize}
        \item Transaction is for internal use while Payment is external
        \item This separates the intent from the execution.
        \item The Transaction is the intent whereas Payment is the execution.
        \item This allows the system to log a failed payment attempt in the Payment table without invalidating the Transaction record, which could then be retried.
    \end{itemize}

\end{enumerate}
\newpage    
\section{Entity-Relation Model}
\label{sec:third}

\subsection{ER Diagram}
\label{subsec:another}
Below is the Entity-Relationship Diagram (ERD) illustrating the structure and relationships between the core data entities in the metro ticket booking system database, including support for both pre-booked tickets and smart card 'tap-and-go' journeys.

\includegraphics[scale= 0.3]{erf2.png}

\subsection{Entities}
\label{subsec:Explanation}

\begin{enumerate}
    \item User
    \begin{itemize}
        \item Represents the system's passenger.
        \item UserID (PK): A unique identifier for each user.
        \item Name, Email, Phone: Personal and contact information.
        \item PasswordHash: Securely stores the user's password instead of plaintext.
        \item CreatedAt, UpdatedAt: Stores when the user first registered and last profile update respectively.
    \end{itemize}

    \item SmartCard
    \begin{itemize}
        \item Represents a metro card linked to a user or used anonymously.
        \item CardID (PK): Unique identifier for the card.
        \item UserID (FK): Links the card to a registered User. 
        \item Balance: The current monetary value stored on the card.
    \end{itemize}

    \item Station
    \begin{itemize}
        \item Represents a physical metro station.
        \item StationID (PK): Unique identifier for the station.
        \item StationName, StationCode (unique): Human-readable names and unique short-codes (e.g., "CEN").
    \end{itemize}

    \item Line
    \begin{itemize}
        \item Represents a single metro line or route.
        \item LineID (PK): Unique identifier for the line.
        \item LineColor: Associated color.
    \end{itemize}

    \item Transaction
    \begin{itemize}
        \item A central "base" entity for all financial activities.
        \item TransactionID (PK): A unique identifier for every transaction.
        \item UserID (FK): Links to the User who initiated the transaction.
        \item Amount: The monetary value of the transaction.
        \item TransactionType: Differentiates the kind of transaction (e.g., "Booking", "Recharge"). This supports the inheritance model.
        \item Status: Success or Failure
    \end{itemize}

    \item Booking
    \begin{itemize}
        \item A specific type of Transaction for purchasing a ticket.
        \item BookingID (PK): Unique identifier for the booking.
        \item TransactionID (FK, unique): A one-to-one link to the parent Transaction record.
        \item SourceStationID (FK), DestinationStationID (FK): Links to the Station table for the start and end of the trip.
        \item TotalFare: The calculated cost for this specific booking.
    \end{itemize}

    \item Recharge
    \begin{itemize}
        \item A specific type of Transaction for adding money to a card.
        \item RechargeID (PK): Unique identifier for the recharge event.
        \item TransactionID (FK, unique): A one-to-one link to the parent Transaction record.
        \item CardID (FK): Links to the SmartCard that is being topped up.
    \end{itemize}

    \item Payment
    \begin{itemize}
        \item Stores the method and status of a Transaction.
        \item PaymentID (PK): Unique identifier for the payment record.
        \item TransactionID (FK, unique): A one-to-one link to the Transaction it paid for.
        \item PaymentMethod: How the payment was made (e.g., "CreditCard", "Wallet", "SmartCardBalance").
        \item Status: Success or Failure
    \end{itemize}

    \item Ticket
    \begin{itemize}
        \item The digital artifact generated by a Booking.
        \item TicketID (PK): Unique identifier for the ticket.
        \item BookingID (FK, unique): A one-to-one link to the Booking that generated this ticket.
        \item QRCodeData (unique): The unique data string for the QR code to be scanned at the gate.
        \item ValidFrom, ValidUntil: The time window during which the ticket can be used.
    \end{itemize}

    \item Journey
    \begin{itemize}
        \item Tracks the actual usage of a Ticket in the metro network.
        \item JourneyID (PK): Unique identifier for the journey.
        \item TicketID (FK, unique): A one-to-one link to the Ticket used for this journey.
        \item EntryStationID (FK), EntryTime: Records which Station the passenger entered and when.
        \item ExitStationID (FK), ExitTime: Records which Station the passenger exited and when. These are nullable because a journey may be in progress.
    \end{itemize}

    \item Fare
    \begin{itemize}
        \item Dictates the exact cost of travel between any two points in the network.
        \item FareID (PK): The unique identifier for each specific fare rule or record.
        \item SourceStationID (FK): Identifies the starting station for which this fare is valid.
        \item DestinationStationID (FK): Identifies the ending station for which this fare is valid.
        \item FareAmount: The precise monetary cost of the journey between the designated source and destination stations.
    \end{itemize}

    \item TJourney
    \begin{itemize}
        \item TJourney is a weak entity set, indicated by a double-lined rectangle.
        \item Its owner entity is Journey. The relationship TJourney\_Journey is its identifying relationship
        \item The relationship Ticket\_TJourney indicates the foreign key constraints with Journey entity set
    \end{itemize}

    \item SCJourney
    \begin{itemize}
        \item SCJourney is a weak entity set, indicated by a double-lined rectangle.
        \item Its owner entity is Journey. The relationship SCJourney\_Journey is its identifying relationship
        \item The relationship Card\_SCJourney indicates the foreign key constraints with SmartCard  entity set
    \end{itemize}

\end{enumerate}

\subsection{Relationships}
\label{subsec:Explanation}

\begin{enumerate}

    \item PaymentTransaction
    \begin{enumerate}
    \renewcommand{\labelenumi}{\roman{enumi}.}
        \item Entities Involved : Payment and Transaction
        \item Relationship Type : One-to-One
        \item Participation Constraints : Total participation for Transaction; Partial participation for Payment
        \item Each Transaction must involve one Payment, and each Payment is associated with at most one Transaction
    \end{enumerate}
    \item UserTransaction
    \begin{enumerate}
    \renewcommand{\labelenumi}{\roman{enumi}.}
        \item Entities Involved : User and Transaction
        \item Relationship Type : One-to-Many
        \item Participation Constraints : Partial participation for User; Partial participation for Transaction
        \item A User can have many Transactions (or none), and a Transaction is performed by one and only one User
    \end{enumerate}

    \item RechargeTransaction
    \begin{enumerate}
    \renewcommand{\labelenumi}{\roman{enumi}.}
        \item Entities Involved : Transaction and Recharge
        \item Relationship Type : One-to-One
        \item Participation Constraints : Total participation for Recharge; Partial participation for SmartCard
        \item A SmartCard can be linked to many Recharges (or none), and every Recharge must be for exactly one SmartCard
    \end{enumerate}

    \item CardRecharge
    \begin{enumerate}
    \renewcommand{\labelenumi}{\roman{enumi}.}
        \item Entities Involved : User and SmartCard
        \item Relationship Type : One-to-Many
        \item Participation Constraints : Total participation for SmartCard; Partial participation for User
        \item A User can possess multiple SmartCards (or none), and every SmartCard is owned by one and only one User
    \end{enumerate}

    \item User\_Card
    \begin{enumerate}
    \renewcommand{\labelenumi}{\roman{enumi}.}
        \item Entities Involved : User and SmartCard
        \item Relationship Type : One-to-Many
        \item Participation Constraints : Total participation for SmartCard; Partial participation for User
        \item A User can possess multiple SmartCards (or none), and every SmartCard is owned by one and only one User
    \end{enumerate}

    \item BookingTransaction
    \begin{enumerate}
    \renewcommand{\labelenumi}{\roman{enumi}.}
        \item Entities Involved : Transaction and Booking
        \item Relationship Type : One-to-One
        \item Participation Constraints : Partial participation for Transaction; Total participation for Booking
        \item Each Booking must be a Transaction, and a Transaction can be a Booking or something else
    \end{enumerate}

    \item TicketBooking
    \begin{enumerate}
    \renewcommand{\labelenumi}{\roman{enumi}.}
        \item Entities Involved : Booking and Ticket
        \item Relationship Type : One-to-One
        \item Participation Constraints : Total participation for Ticket; Total participation for Booking
        \item A Booking corresponds to exactly one Ticket, and every Ticket is created via one Booking
    \end{enumerate}

    \item Ticket\_TJourney 
    \begin{enumerate}
    \renewcommand{\labelenumi}{\roman{enumi}.}
        \item Entities Involved : Ticket and TJourney
        \item Relationship Type : One-to-One
        \item Participation Constraints : Total participation for TJourney; Partial participation for Ticket
        \item A Ticket can be used for one TJourneys, and a TJourney can involve a single ticket
    \end{enumerate}

    \item TJourney\_Journey (Weak)
    \begin{enumerate}
    \renewcommand{\labelenumi}{\roman{enumi}.}
        \item Entities Involved : TJourney and Journey
        \item Relationship Type : One-to-One
        \item Participation Constraints : Total participation for TJourney; Partial participation for Journey
        \item Each TJourney is associated with a Journey, but a Journey may not be associated with TJourney
    \end{enumerate}

    \item SCJourney\_Journey (Weak)
    \begin{enumerate}
    \renewcommand{\labelenumi}{\roman{enumi}.}
        \item Entities Involved : Journey and SCJourney
        \item Relationship Type : One-to-One
        \item Participation Constraints : Total participation for SCJourney; Partial participation for Journey
        \item Each SCJourney is associated with a Journey, but a Journey may not be associated with SCJourney

    \end{enumerate}

    \item EntryStation
    \begin{enumerate}
    \renewcommand{\labelenumi}{\roman{enumi}.}
        \item Entities Involved : Station and Journey
        \item Relationship Type : One-to-Many
        \item Participation Constraints : Total participation for Journey; Partial participation for Station
        \item A Station can be the entry point for many Journeys (or none), and every Journey must have exactly one entry Station
    \end{enumerate}

    \item ExitStation
    \begin{enumerate}
    \renewcommand{\labelenumi}{\roman{enumi}.}
        \item Entities Involved : Station and Journey
        \item Relationship Type : One-to-Many
        \item Participation Constraints : Partial participation for Journey; Partial participation for Station
        \item A Station can be the exit point for many Journeys (or none), and a Journey may have an exit Station (e.g., if the journey is complete)
    \end{enumerate}

    \item StationLine
    \begin{enumerate}
    \renewcommand{\labelenumi}{\roman{enumi}.}
        \item Entities Involved : Station and MetroLine
        \item Relationship Type : Many-to-Many
        \item Participation Constraints : Total participation for Station; Total participation for MetroLine
        \item Station can belong to multiple MetroLines, and a MetroLine consists of multiple Stations
    \end{enumerate}

    \item Card\_SCJourney
    \begin{enumerate}
    \renewcommand{\labelenumi}{\roman{enumi}.}
        \item Entities Involved : SmartCard and SCJourney
        \item Relationship Type : One-to-Many
        \item Participation Constraints : Total participation for SCJourney; Partial participation for SmartCard
        \item A SmartCard can be linked to many SCJourneys, and every SCJourney involves one SmartCard
    \end{enumerate}

    \item To
    \begin{enumerate}
    \renewcommand{\labelenumi}{\roman{enumi}.}
        \item Entities Involved : Station and Fare
        \item Relationship Type : One-to-Many
        \item Participation Constraints : Total participation for Fare, Partial participation for Station
        \item A Station can be the destination for many Fare records, and every Fare is defined for one destination Station
    \end{enumerate}

    \item From
    \begin{enumerate}
    \renewcommand{\labelenumi}{\roman{enumi}.}
        \item Entities Involved : Station and Fare
        \item Relationship Type : One-to-Many
        \item Participation Constraints : Total participation for Fare : Partial participation for Station
        \item A Station can be the origin for many Fare records, and every Fare is defined from one origin Station
    \end{enumerate}
\end{enumerate}


\newpage
\section{Schema}
\label{sec:schema}

The following tables define the schema for the Delhi Metro Ticket Booking System.
\setcounter{table}{0}

% -------------------------
% USER TABLE
% -------------------------
\begin{table}[h!]
\centering
\begin{tabular}{|l|l|l|}
\hline
\rowcolor{tableheader}
\textbf{Attribute} & \textbf{Type} & \textbf{Description} \\ \hline
UserID & INT (PK) & Unique identifier for each user \\ \hline
Name & VARCHAR(100) & Full name of the user \\ \hline
Email & VARCHAR(100) & Email address (unique) \\ \hline
Phone & VARCHAR(15) & Contact number \\ \hline
PasswordHash & VARCHAR(255) & Encrypted password \\ \hline
\end{tabular}
\caption{User}
\end{table}

% -------------------------
% SMARTCARD TABLE
% -------------------------
\begin{table}[h!]
\centering
\begin{tabular}{|l|l|l|}
\hline
\rowcolor{tableheader}
\textbf{Attribute} & \textbf{Type} & \textbf{Description} \\ \hline
\rowcolor{rowcolor}
CardID & INT (PK) & Unique identifier for each card \\ \hline
\rowcolor{rowcolor}
UserID & INT (FK) & Linked user ID (nullable) \\ \hline
\rowcolor{rowcolor}
Balance & DECIMAL(10,2) & Current stored balance \\ \hline
\end{tabular}
\caption{SmartCard}
\end{table}

% -------------------------
% STATION TABLE
% -------------------------

\begin{table}[h!]
\centering
\begin{tabular}{|l|l|l|}
\hline
\rowcolor{tableheader}
\textbf{Attribute} & \textbf{Type} & \textbf{Description} \\ \hline
\rowcolor{rowcolor}
StationID & INT (PK) & Unique identifier for each metro station \\ \hline
\rowcolor{rowcolor}
StationName & VARCHAR(100) & Human-readable name of the station \\ \hline
\rowcolor{rowcolor}
StationCode & VARCHAR(10) & Unique short code (e.g., "CEN") \\ \hline
\end{tabular}
\caption{Station}
\end{table}

% -------------------------
% LINE TABLE
% -------------------------

\begin{table}[h!]
\centering
\begin{tabular}{|l|l|l|}
\hline
\rowcolor{tableheader}
\textbf{Attribute} & \textbf{Type} & \textbf{Description} \\ \hline
\rowcolor{rowcolor}
LineID & INT (PK) & Unique identifier for each metro line \\ \hline
\rowcolor{rowcolor}
LineName & VARCHAR(100) & Human-readable name (e.g., "Blue Line") \\ \hline
\rowcolor{rowcolor}
LineColor & VARCHAR(30) & Color representing the line (e.g., "Blue") \\ \hline
\end{tabular}
\caption{Line}
\end{table}

% -------------------------
% TRANSACTION TABLE
% -------------------------
\rowcolors{2}{rowcolor}{white}
\begin{table}[h!]
\centering
\begin{tabular}{|l|l|l|}
\hline
\rowcolor{tableheader}
\textbf{Attribute} & \textbf{Type} & \textbf{Description} \\ \hline
TransactionID & INT (PK) & Unique transaction identifier \\ \hline
UserID & INT (FK) & User who initiated transaction \\ \hline
Amount & DECIMAL(10,2) & Value of the transaction \\ \hline
TransactionType & VARCHAR(20) & Booking / Recharge \\ \hline
Timestamp & DATETIME & Time of transaction \\ \hline
\end{tabular}
\caption{Transaction}
\end{table}


% -------------------------
% BOOKING TABLE
% -------------------------
\rowcolors{2}{rowcolor}{white}
\begin{table}[h!]
\centering
\begin{tabular}{|l|l|l|}
\hline
\rowcolor{tableheader}
\textbf{Attribute} & \textbf{Type} & \textbf{Description} \\ \hline
BookingID & INT (PK) & Unique booking identifier \\ \hline
TransactionID & INT (FK, unique) & Parent transaction \\ \hline
SourceStationID & INT (FK) & Starting station \\ \hline
DestinationStationID & INT (FK) & Ending station \\ \hline
TotalFare & DECIMAL(10,2) & Fare for this booking \\ \hline
\end{tabular}
\caption{Booking}
\end{table}

% -------------------------
% RECHARGE TABLE
% -------------------------
\rowcolors{2}{rowcolor}{white}
\begin{table}[h!]
\centering
\begin{tabular}{|l|l|l|}
\hline
\rowcolor{tableheader}
\textbf{Attribute} & \textbf{Type} & \textbf{Description} \\ \hline
RechargeID & INT (PK) & Unique recharge ID \\ \hline
TransactionID & INT (FK, unique) & Parent transaction \\ \hline
CardID & INT (FK) & Card being recharged \\ \hline
\end{tabular}
\caption{Recharge}
\end{table}

% -------------------------
% PAYMENT TABLE
% -------------------------
\rowcolors{2}{rowcolor}{white}
\begin{table}[h!]
\centering
\begin{tabular}{|l|l|l|}
\hline
\rowcolor{tableheader}
\textbf{Attribute} & \textbf{Type} & \textbf{Description} \\ \hline
PaymentID & INT (PK) & Unique payment ID \\ \hline
TransactionID & INT (FK, unique) & Transaction being paid \\ \hline
PaymentMethod & VARCHAR(50) & Method of payment \\ \hline
Status & VARCHAR(20) & Payment status (Success/Failed) \\ \hline
\end{tabular}
\caption{Payment}
\end{table}

% -------------------------
% TICKET TABLE
% -------------------------
\rowcolors{2}{rowcolor}{white}
\begin{table}[h!]
\centering
\begin{tabular}{|l|l|l|}
\hline
\rowcolor{tableheader}
\textbf{Attribute} & \textbf{Type} & \textbf{Description} \\ \hline
TicketID & INT (PK) & Unique ticket ID \\ \hline
BookingID & INT (FK, unique) & Linked booking \\ \hline
QRCodeData & VARCHAR(255) & Unique QR string \\ \hline
ValidFrom & DATETIME & Start of validity \\ \hline
ValidUntil & DATETIME & End of validity \\ \hline
\end{tabular}
\caption{Ticket}
\end{table}

% -------------------------
% JOURNEY TABLE
% -------------------------
\rowcolors{2}{rowcolor}{white}
\begin{table}[h!]
\centering
\begin{tabular}{|l|l|l|}
\hline
\rowcolor{tableheader}
\textbf{Attribute} & \textbf{Type} & \textbf{Description} \\ \hline
JourneyID & INT (PK) & Unique journey ID \\ \hline
JourneyType & ENUM & Ticket or Card used \\ \hline
EntryStationID & INT (FK) & Entry station \\ \hline
EntryTime & DATETIME & Time of entry \\ \hline
ExitStationID & INT (FK) & Exit station (nullable) \\ \hline
ExitTime & DATETIME & Time of exit (nullable) \\ \hline
FareDeducted & INT & Actual amount (nullable) \\ \hline
Status & ENUM & Journey status \\ \hline
\end{tabular}
\caption{Journey}
\end{table}

% -------------------------
% FARE TABLE
% -------------------------
\rowcolors{2}{rowcolor}{white}
\begin{table}[h!]
\centering
\begin{tabular}{|l|l|l|}
\hline
\rowcolor{tableheader}
\textbf{Attribute} & \textbf{Type} & \textbf{Description} \\ \hline
FareID & INT (PK) & Unique fare record \\ \hline
SourceStationID & INT (FK) & Start station \\ \hline
DestinationStationID & INT (FK) & End station \\ \hline
FareAmount & DECIMAL(10,2) & Fare value \\ \hline
\end{tabular}
\caption{Fare}
\end{table}

\newpage

\section{Normalisation}

All tables satisfy 1NF because all attributes (like INT, VARCHAR, DECIMAL) are atomic, and there are no repeating groups or multi-valued attributes.

\begin{enumerate}
    \item User
    \begin{itemize}
        \item Primary Key (PK): UserID
        \item Candidate Keys (CKs): UserID, Email (since it's unique)
        \item 3NF: Yes. The table is in 2NF (PK is a single attribute). It has no transitive dependencies; non-key attributes like Name and Phone depend only on the primary key, not on another non-key attribute.
    \end{itemize}

    \item SmartCard
    \begin{itemize}
        \item Primary Key (PK): CardID
        \item Candidate Keys (CKs):  CardID
        \item 3NF: Yes. It's in 2NF (single-attribute PK). It has no transitive dependencies; Balance and UserID depend only on the CardID.
    \end{itemize}

    \item Station
    \begin{itemize}
        \item Primary Key (PK): StationID
        \item Candidate Keys (CKs):  StationID, StationCode (since it's unique)
        \item 3NF:  Yes. It's in 2NF (single-attribute PK). It has no transitive dependencies; StationName depends directly on the key, not on another non-key attribute.
    \end{itemize}

    \item Line
    \begin{itemize}
        \item Primary Key (PK): LineID
        \item Candidate Keys (CKs):  LineID
        \item 3NF: Yes. It's in 2NF (single-attribute PK). It has no transitive dependencies; LineName and LineColor depend only on LineID.
    \end{itemize}

    \item Transaction
    \begin{itemize}
        \item Primary Key (PK): TransactionID
        \item Candidate Keys (CKs):  TransactionID
        \item 3NF: Yes. It's in 2NF (single-attribute PK). It has no transitive dependencies; all other attributes (UserID, Amount, etc.) depend only on TransactionID.
    \end{itemize}

    \item Booking
    \begin{itemize}
        \item Primary Key (PK): BookingID
        \item Candidate Keys (CKs):  BookingID, TransactionID (since it's unique)
        \item 3NF: Yes. It's in 2NF (single-attribute PK). It has no transitive dependencies. Attributes like TotalFare and SourceStationID depend directly on the key (BookingID or TransactionID), not on another non-key attribute.
    \end{itemize}

    \item Recharge
    \begin{itemize}
        \item Primary Key (PK): RechargeID
        \item Candidate Keys (CKs):  RechargeID, TransactionID (since it's unique)
        \item 3NF: Yes. It's in 2NF (single-attribute PK). It has no transitive dependencies; CardID depends directly on the key.
    \end{itemize}

    \item Payment
    \begin{itemize}
        \item Primary Key (PK): PaymentID
        \item Candidate Keys (CKs):  PaymentID, TransactionID (since it's unique)
        \item 3NF: Yes. It's in 2NF (single-attribute PK). It has no transitive dependencies; PaymentMethod and Status depend only on the key.
    \end{itemize}

    \item Ticket
    \begin{itemize}
        \item Primary Key (PK): TicketID
        \item Candidate Keys (CKs):  TicketID, BookingID (unique), QRCodeData (unique)
        \item 3NF: Yes. It's in 2NF (single-attribute PK). It has no transitive dependencies; all other attributes depend directly on one of the candidate keys.
    \end{itemize}

    \item Journey
    \begin{itemize}
        \item Primary Key (PK): JourneyID
        \item Candidate Keys (CKs):  JourneyID
        \item 3NF: Yes. It's in 2NF (single-attribute PK). It has no transitive dependencies; all other attributes, like EntryStationID, EntryTime, ExitStationID, etc., depend only on the JourneyID.
    \end{itemize}

    \item Fare
    \begin{itemize}
        \item Primary Key (PK): FareID
        \item Candidate Keys (CKs):  FareID, and the composite key (SourceStationID, DestinationStationID) (assuming a single fare per route).
        \item 3NF: Yes. It's in 2NF (single-attribute PK). It has no transitive dependencies. The dependency (SourceStationID, DestinationStationID) -> FareAmount is valid because the determinant is a candidate key.
    \end{itemize}
\end{enumerate}
% ----------------------------------------------------------------------------
% END DOCUMENT
% ----------------------------------------------------------------------------
\end{document}